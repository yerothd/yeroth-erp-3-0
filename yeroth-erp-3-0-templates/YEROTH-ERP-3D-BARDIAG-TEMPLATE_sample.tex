\def\onecol{forestgreen}
\def\onetopcol{firebrickred}
\def\twocol{forestgreen}
\def\twotopcol{firebrickred}
\def\threecol{forestgreen}
\def\threetopcol{firebrickred}
\def\fourcol{forestgreen}
\def\fourtopcol{firebrickred}
\def\fivecol{forestgreen}
\def\fivetopcol{firebrickred}
\def\sixcol{forestgreen}
\def\sixtopcol{firebrickred}
\def\sevencol{forestgreen}
\def\seventopcol{firebrickred}
%-------------------------------------------------------
%This is the way to redefine styles
% \newpsstyle{mytickstyle}{linewidth=1pt,linecolor=blue}
%
% Style of the foreground frame
\newpsstyle{diagframestyle}{linewidth=1pt,linecolor=black,fillcolor=white}
% Style of the background frame
\newpsstyle{diagbgframestyle}{linewidth=1pt,linecolor=black,fillcolor=yerothColorGray}

% Use 3D bars
\renewcommand{\ActiveBarPrimitive}{\barTDRect}
% Make frame 3D
\renewcommand{\frameTD}{1}

% Put ticks and levellines each 100 yunits
\renewcommand{\betweenticks}{100}

% Color of the numbers on the bar-items 
\renewcommand\numbercolor{\black\bf}
% Where to put the number. Can be \bottom,\belowtop,\overtop
\renewcommand{\placenumber}{\bottom}

\renewcommand{\tdx}{1.2} % depth of 3d
\renewcommand{\tdy}{6}

% Start the diagram
% #1 diagram height (number)
% #2 diagram width (number)
% #3 bottom height (length, use cm, in,. . . )
% #4 bar width (number)
% #5 distance between (groups) of bars
% #6 x unit length (cm, in,. . . )
% #7 y unit length (cm, in,. . . )
\bardiagrambegin{18.9}{900}{2cm}{1}{5}{0.8cm}{0.01cm}
       \baritem{LUNDI}{32}{\onecol}
          \subtopbaritem{}{40}{\onetopcol}
        \subbaritem{MARDI}{20}{\twocol}
          \subtopbaritem{}{30}{\twotopcol}
        \subbaritem{MERCREDI}{13}{\threecol}
          \subtopbaritem{}{50}{\threetopcol}
       %---
       \baritem{LUNDI}{21}{\onecol}
          \subtopbaritem{}{60}{\onetopcol}
        \subbaritem{MARDI}{25}{\twocol} 
          \subtopbaritem{}{64}{\twotopcol}
        \subbaritem{MERCREDI}{58}{\threecol}  
          \subtopbaritem{}{32}{\threetopcol}
       %---
       \baritem{LUNDI}{22}{\onecol}
          \subtopbaritem{}{19}{\onetopcol}
        \subbaritem{MARDI}{12}{\twocol}
          \subtopbaritem{}{18}{\twotopcol}
        \subbaritem{MERCREDI}{9}{\threecol} 
          \subtopbaritem{}{12}{\threetopcol}
\drawlevellines
    % Legend
    %  Let's make the background white
    %  and gray frame-line of 0.5pt
    \diagLegendoptions{white}{gray}{0.5pt}
    %
    %\renewcommand{\legendShadowColor}{yerothColorGray}
    %
    \diagLegendbegin{12}{882}{9}
      \diagLegenditem{YEROTHBASELEGEND}{\onecol}
      \diagLegenditem{YEROTHTOPLEGEND}{\onetopcol}
    \diagLegendend
    % End of the legend
\setlength{\captionoffset}{2cm}
\bardiagramend{\parbox{11cm}{{\centering \vspace{2.5cm} YEROTHX\\[0.3cm] 
   \hspace{0cm}
   \begin{tabular}{p{3.2cm}p{4.0cm}p{3.2cm}}
      \centering YEROTHONE & \centering YEROTHTWO & \centering YEROTHREE
   \end{tabular}
   \vspace{-0.3em}
  }}}
  {\large YEROTHZ}
  \vspace{3em}
  
  
